% This is the ADASS_template.tex LaTeX file, 26th August 2016.
% It is based on the ASP general author template file, but modified to reflect the specific
% requirements of the ADASS proceedings.
% Copyright 2014, Astronomical Society of the Pacific Conference Series
% Revision:  14 August 2014

% To compile, at the command line positioned at this folder, type:
% latex ADASS_template
% latex ADASS_template
% dvipdfm ADASS_template
% This will create a file called aspauthor.pdf.}

\documentclass[11pt,twoside]{article}

% Do NOT use ANY packages other than asp2014. 
\usepackage{asp2014}

\aspSuppressVolSlug
\resetcounters

% References must all use BibTeX entries in a .bibfile.
% References must be cited in the text using \citet{} or \citep{}.
% Do not use \cite{}.
% See ManuscriptInstructions.pdf for more details
\bibliographystyle{asp2014}

% The ``markboth'' line sets up the running heads for the paper.
% 1 author: "Surname"
% 2 authors: "Surname1 and Surname2"
% 3 authors: "Surname1, Surname2, and Surname3"
% >3 authors: "Surname1 et al."
% Replace ``Short Title'' with the actual paper title, shortened if necessary.
% Use mixed case type for the shortened title
% Ensure shortened title does not cause an overfull hbox LaTeX error
% See ASPmanual2010.pdf 2.1.4  and ManuscriptInstructions.pdf for more details
\markboth{Peterson et al.}{Sky Viewer Investigations for LSST EPO}

\begin{document}

\title{Sky Viewer Investigations for LSST EPO}

% Note the position of the comma between the author name and the 
% affiliation number.
% Author names should be separated by commas.
% The final author should be preceded by "and".
% Affiliations should not be repeated across multiple \affil commands. If several
% authors share an affiliation this should be in a single \affil which can then
% be referenced for several author names.
% See ManuscriptInstructions.pdf and ASPmanual2010.pdf 3.1.4 for more details
\author{J.~Matt~Peterson,$^1$ Amanda~E.~Bauer,$^1$ Ben~Emmons,$^1$ Ryan~Bruner,$^2$ Jimmy~Watkins,$^2$ Tom~Gonzalez,$^2$ and Paula~Wellings$^2$}
\affil{$^1$Large Synoptic Survey Telescope, Tucson, AZ, USA}
\affil{$^2$Theresa Neil: Strategy + Design, Austin, TX, USA}

% This section is for ADS Processing.  There must be one line per author.
\paperauthor{J.~Matt~Peterson}{jmatt@lsst.org}{http://orcid.org/0000-0002-6564-6246}{Large Synoptic Survey Telescope}{}{Tucson}{AZ}{85719}{USA}
\paperauthor{Amanda~E.~Bauer}{abauer@lsst.org}{}{Large Synoptic Survey Telescope}{}{Tucson}{AZ}{85719}{USA}
\paperauthor{Ben~Emmons}{bemmons@lsst.org}{}{Large Synoptic Survey Telescope}{}{Tucson}{AZ}{85719}{USA}
\paperauthor{Ryan~Bruner}{ryan@strategyplusdesign.com}{}{Theresa Neil: Strategy + Design}{}{Austin}{TX}{}{USA}
\paperauthor{Jimmy~Watkins}{jimmy.what@gmail.com}{}{Theresa Neil: Strategy + Design}{}{Austin}{TX}{}{USA}
\paperauthor{Tom~Gonzalez}{tgonzalez@brightpointinc.com}{}{Theresa Neil: Strategy + Design}{}{Austin}{TX}{}{USA}
\paperauthor{Paula~Wellings}{paula@strategyplusdesign.com}{}{Theresa Neil: Strategy + Design}{}{Austin}{TX}{}{USA}

\begin{abstract}
The Large Synoptic Survey Telescope (LSST) is an 8-m optical ground-based telescope being constructed on Cerro Pachon in Chile. LSST will survey half the sky every few nights in six optical bands. LSST Education and Public Outreach (EPO) is investigating requirements and technologies to create a next generation sky viewer. We discuss and survey current sky viewer technologies and identify new features required for an LSST EPO Skyviewer.
\end{abstract}

\section{Introduction}
LSST Education and Public Outreach (EPO) reports on the early investigations of using and implementing a sky viewer across its software systems. We contracted with Theresa Neil: Strategy + Design (TNSD)\footnotemark[1] to develop a proof of concept sky viewer to test with a variety of intended users of the EPO Portal in order to understand and achieve the needs of our sky viewer audiences. We researched available open-source sky viewer software to identify software engineer requirements and features for our implementation \citep{2017LSST.1.LSE-89,2017LSST.1.LEP-31}. Over the summer of 2017 we prototyped an implementation of this using Aladin and Aladin-lite \citep{2014ASPC..485..277B} and evaluated other available tools.

\footnotetext[1]{\url{http://www.theresaneil.com/}}

\section{Requirements}
The LSST EPO Skyviewer has three software engineering requirements:
\begin{enumerate}
\item Built on an existing sustainable open-source projects.
\item Simple to extend and integrate with existing tools and libraries.
\item Usable in multiple contexts (e.g. mobile, web, science notebooks, planetarium).
\end{enumerate}

\section{Implementation}
Aladin v10.009 and Hipsgen were used to create a prototype survey and sky viewer with HSC Public Data Release 1 UDEEP COSMOS data (2 sq. deg.) \citep{2017arXiv170208449A}. This was reprocessed by the LSST Data Management (DM) team using the LSST DM pipeline \citep{2015arXiv151207914J}. Hierarchical Progressive Survey \citep[HiPS,][]{2015A&A...578A.114F} creation took 16 hours on a single node using 8 vCPUs and 16 GB of RAM to process 127 GB of calibrated exposure FITS images across three bands (\emph{i}, \emph{r}, \emph{g}) into a 16.7 GB color PNG image HiPS (HiPS Norder=3-12). To create multicolor HiPS individual monocolor HiPS must first be created. From these monocolor HiPS a multicolor HiPS is then composed. This method increases linearly with disk usage and processing time for each band being processed. This is in contrast to other survey creation methods which do not require the creation of intermediate monocolor surveys.
\section{Comparison}
\begin{table}[!ht]
\begin{center}
{\small
 \begin{tabular}{|l|l|l|l|l|}  % l = left, c = centered
\hline
~ & \begin{tabular}{@{}l@{}}Aladin \& \\ Aladin-lite\end{tabular}
  & \begin{tabular}{@{}l@{}}Leaflet based \\ sky viewers\end{tabular}
  & \begin{tabular}{@{}l@{}}World \\ Wide \\ Telescope (WWT)\end{tabular} \\
\hline
Projection & HEALPix & Mercator & TOAST \\
\hline
Tile Manager & HiPS & Leaflet & WWT \\
\hline
Tile Creation & Aladin & ImageMagick, custom & Montage \\
\hline
Examples
& \begin{tabular}{@{}l@{}}ESA, CDS,\footnotemark[2] \\ DES,\footnotemark[3] LIGO\footnotemark[4]\end{tabular}
& \begin{tabular}{@{}l@{}}DECaLS,\footnotemark[5] CFHTLS,\footnotemark[6] \\ HSC-SSP,\footnotemark[7] NASA, SDSS\footnotemark[8]\end{tabular}
    & WWT web client \\
    \hline
\end{tabular}
  }
\end{center}
\caption{Comparison of three types of sky viewers.}
\label{table:p140_comparison}
\end{table}
\footnotetext[2]{Centre de Donn\~{A}es astronomiques de Strasbourg}
\footnotetext[3]{Dark Energy Survey}
\footnotetext[4]{Laser Interferometer Gravitational-Wave Observatory}
\footnotetext[5]{DECam Legacy Survey}
\footnotetext[6]{Canada-France-Hawaii Telescope Legacy Survey}
\footnotetext[7]{Hyper Suprime-Cam Subaru Strategic Program}
\footnotetext[8]{Sloan Digital Sky Survey}

\noindent Aladin and Aladin Lite use HEALPix \citep{2007MNRAS.381..865C}. HEALPix is supported in multiple libraries in Python 3. Python 3 has seen extensive use at LSST \citep{P1-123_adassxxvii} and is used in our Jupyter \citep{PER-GRA:2007} science notebook investigations. HiPS and  Multi-Order Coverage map \citep[MOC,][]{2015A&A...578A.114F}, which use HEALPix, are International Virtual Observatory Alliance (IVOA) standards we plan to support. Finally, HEALPix is acceptable for use with the LSST science platform \citep{2017LSST.1.LDM-542} and our formal education teams.

Montage \citep{2017PASP..129e8006B} was evaluated to create Tessellated Octahedral Adaptive Subdivision Transform (TOAST) for the World Wide Telescope (WWT) \citep{2012ASPC..461..267G} and Leaflet based surveys. Montage supports HEALPix FITS files as input data but it does not create HiPS. WWT, TOAST and Montage were not chosen for two reasons. First, Montage does not create HiPS. Second, TOAST has no viable stand-alone viewer and is only available as part of WWT as a monolithic application. This makes it difficult to customize and provide a sky viewer that is usable in multiple contexts. LSST EPO is interested in WWT support for data cubes, 3D and virtual reality.
\section{User Interface}
\begin{figure}[!ht]
  \plotone{P1-140_f1.eps}
  \caption{Planned design mockup features for the LSST EPO Skyviewer. Features are labeled as Gallery(1), Object slider(2), Detail slider(3), Search(4), Filter(5) and Collections(6).}
  \label{P1-140_fig1}
\end{figure}
\noindent EPO contracted design work and user experience testing from TNSD. Through user experience testing, TNSD identified improvements with existing sky viewer design and iteratively improved the design. Users found sky viewer use was not intuitive and simple. This was resolved by adding a gallery panel displaying objects in view and a search function. Users wanted information about the object being viewed. This was resolved by adding the object and detail sliders. Finally, users wanted a way to filter and save what they viewed. This was resolved by adding filters and collections. Figure \ref{P1-140_fig1} is the result of this iterative design and user testing.

%% \articlefigure{P1-140_f1.eps}{P1-140_fig1}{Planned design mockup features for the LSST Skyviewer.}

%% \begin{table}[!ht]
%%   \small {
%%     \begin{tabular}{ll}
%%       \begin{tabular}{@{}l@{}}1. Gallery \\ 2. Object slider \\ 3. Detail slider \end{tabular}
%%       & \begin{tabular}{@{}l@{}}4. Search \\ 5. Filter \\ 6. Collections \end{tabular}
%%     \end{tabular}
%%   }
%% \end{table}

\section{Conclusion}
EPO plans to implement the LSST EPO Skyviewer as modular open-source software components. These components will build on IVOA standards and should be reusable for different audiences through configuration and extension. Components will be used by the EPO Portal and science notebook systems. Since our science notebook system will extend the LSST science platform, our components will be available to LSST scientists.

We plan to use Aladin Lite using HiPS. Improvements in either Hipsgen or Montage to render HiPS more efficiently are necessary. Additionally, WWT using TOAST may still be considered as an alternative if there is a stand-alone viewer.

\acknowledgements This material is based upon work supported in part by the Na- tional Science Foundation through Cooperative Agreement 1258333 managed by the Association of Universities for Research in Astronomy (AURA), and the Department of Energy under Contract No. DE-AC02-76SF00515 with the SLAC National Accel- erator Laboratory. Additional LSST funding comes from private donations, grants to universities, and in-kind support from LSSTC Institutional Members.

We thank Tim Jenness for encouraging us to publish our investigations.

\bibliography{P1-140}  % For BibTex

\end{document}
